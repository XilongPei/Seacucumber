%Hier sind alle Einstellungen enthalten, die sich auf das Seiten- und
%Dokumentenlayout beziehen

\documentclass[
  11pt,                   % Schriftgröße
  DIV12,
  german,                 % für Umlaute, Silbentrennung etc.
  oneside,                % einseitiges Dokument
  titlepage,              % es wird eine Titelseite verwendet
  parskip=half,           % Abstand zwischen Absätzen (halbe Zeile)
  headings=normal,        % Größe der Überschriften verkleinern
  captions=tableheading,  % Beschriftung von Tabellen unterhalb ausgeben
  final                   % Status des Dokuments (final/draft)
]{scrreprt}               %


%------Ändern von Schriftschnitten - (Muss ganz am Anfang stehen !) ------------
\usepackage{fix-cm}


%------Umlaute -----------------------------------------------------------------
%   Umlaute/Sonderzeichen wie äüöß können direkt im Quelltext verwenden werden.
%    Erlaubt automatische Trennung von Worten mit Umlauten.
\usepackage[T1]{fontenc}
\usepackage[utf8]{inputenc}

%------Anpassung der Landessprache----------------------------------------------
\usepackage[ngerman]{babel}

%------Einfache Definition der Zeilenabstände und Seitenränder------------------
\usepackage{geometry}
\usepackage{setspace}

%------Schriftgrößenanpassung von einzelnen Textpassagen------------------------
\usepackage{relsize}

%------Trennlinien in Kopf- und Fusszeile
\usepackage[headsepline, footsepline, ilines]{scrpage2}

%------Grafiken und Farben -----------------------------------------------------
\usepackage{xcolor}
\usepackage{graphicx}

%------Packet zum Sperren, Unterstreichen und Hervorheben von Texten------------
\usepackage{soul}

%------ergänzende Schriftart----------------------------------------------------
\usepackage{helvet}

%------Lange Tabellen-----------------------------------------------------------
\usepackage{longtable}
\usepackage{array}
\usepackage{ragged2e}
\usepackage{lscape}

%------PDF-Optionen-------------------------------------------------------------
\usepackage[
  bookmarks,
  bookmarksopen=true,
  colorlinks=true,
  linkcolor=black,        % einfache interne Verknüpfungen
  anchorcolor=black,      % Ankertext
  citecolor=black,        % Verweise auf Literaturverzeichniseinträge im Text
  filecolor=black,        % Verknüpfungen, die lokale Dateien öffnen
  menucolor=black,        % Acrobat-Menüpunkte
  urlcolor=black,         % Farbe für URL-Links
  backref,                % Zurücktext nach jedem Bibliografie-Eintrag als
                          % Liste von Überschriftsnummern
  pagebackref,            % Zurücktext nach jedem Bibliografie-Eintrag als
                          % Liste von Seitenzahlen
  plainpages=false,       % zur korrekten Erstellung der Bookmarks
  pdfpagelabels,          % zur korrekten Erstellung der Bookmarks
  hypertexnames=false,    % zur korrekten Erstellung der Bookmarks
  linktocpage             % Seitenzahlen anstatt Text im Inhaltsverzeichnis verlinken
  ]{hyperref}






      % enthält eingebundene Packete

%------Seitenränder-------------------------------------------------------------
\geometry{verbose,                     % zeigt die eingestellten Parameter beim
                                       % Latexlauf an
      paper=a4paper,                   % Papierformat
      top=25mm,                        % Rand oben
      left=25mm,                       % Rand links
      right=25mm,                      % Rand rechts
      bottom=45mm,                     % Rand unten
      pdftex                           % schreibt das Papierformat in die
                                       % Ausgabe damit Ausgabeprogramm
                                       % Papiergröße erkennt
  }

%Seitenlayout
\onehalfspace        % 1,5-facher Abstand

%------Kopf- und Fußzeilen -----------------------------------------------------
\pagestyle{scrheadings}

%------Kopf- und Fußzeile auch auf Kapitelanfangsseiten ------------------------
\renewcommand*{\chapterpagestyle}{scrheadings}

%------Schriftform der Kopfzeile -----------------------------------------------
\renewcommand{\headfont}{\normalfont}

%----Spezielle Befehle
\newcommand{\lfk}[1]{$\langle LF#1\rangle$}

%----Farben
\definecolor{tubsRed}{cmyk}{0.1,1.0,0.8,0.0}
\definecolor{tuRed}{cmyk}{0.1,1.0,0.8,0.0}
\definecolor{dkgreen}{rgb}{0,0.6,0}
\definecolor{gray}{rgb}{0.5,0.5,0.5}
%\definecolor{mauve}{rgb}{0.58,0,0.82}

%------Kopfzeile----------------------------------------------------------------
\setheadsepline{1pt}[\color{tuRed}]
\setlength{\headheight}{21mm}        % Höhe der Kopfzeile
\ihead{\large{\textsc{\praktikumTitel}}\\    % Text in der linken Box
       \small{\projektTitel}}
\chead{}                            % Text in der mittleren Box

%----Fusszeile
\setfootsepline{1pt}[\color{tuRed}]
\cfoot{}                            % Text in mittlerer Box
\ofoot{\pagemark}                    % Seitenzahl in rechter Box



%------Labels mit eigenem Text für \ref ----------------------------------------
\makeatletter
\def\namedlabel#1#2{\begingroup
#2%
\def\@currentlabel{#2}%
\phantomsection\label{#1}\endgroup
}
\makeatother


%------Neue Environments -------------------------------------------------------

\newcommand{\refsetcounter}[2]{\setcounter{#1}{#2}\addtocounter{#1}{-1}\refstepcounter{#1}}

%Funktion im Pflichtenheft
\newcounter{functioncount} 
\newenvironment{function}[2]{\refsetcounter{functioncount}{#1}\large\textbf{\sffamily{#2 }}\namedlabel{F#1}{$\langle F#1\rangle$}\normalsize\begin{description}\setlength{\itemsep}{-5pt}}{\end{description}}

%Daten im Pflichtenheft
\newcounter{datacount} 
\newenvironment{data}[2]{\refsetcounter{datacount}{#1}\textbf{#2} \namedlabel{D#1}{$\langle D#1\rangle$}\\}{}

%Kriterien im Pflichtenheft
\newcounter{mustcount} 
\newcommand{\must}[2]{\refsetcounter{mustcount}{#1}\namedlabel{RM#1}{$\langle RM#1\rangle$} #2\\}

\newcommand{\should}[2]{\refsetcounter{datacount}{#1}\namedlabel{RS#1}{$\langle RS#1\rangle$} #2\\}

\newcommand{\could}[2]{\refsetcounter{datacount}{#1}\namedlabel{RC#1}{$\langle RC#1\rangle$} #2\\}

\newcommand{\wont}[2]{\refsetcounter{datacount}{#1}\namedlabel{RW#1}{$\langle RW#1\rangle$} #2\\}

%Qualitätsanforderungen im Pflichtenheft
\newcommand{\qualityReq}[2]{\refsetcounter{datacount}{#1}\namedlabel{Q#1}{$\langle Q#1\rangle$} #2\\}

% Benutzeroberflächen im Pflichtenheft
\newcounter{uicount}
\newenvironment{ui}[2]{\refsetcounter{uicount}{#1}\textbf{#2} \namedlabel{UI#1}{$\langle UI#1\rangle$}\\}{}

% Klassen
\newcounter{classcount}
\newenvironment{class}[2]{\refsetcounter{classcount}{#1}\textbf{#2}\namedlabel{CL#1}{$\langle CL#1\rangle$}\begin{description}\setlength{\itemsep}{-5pt}}{\end{description}}

% Entitäten
\newcounter{entitycount}
\newenvironment{entity}[2]{\refsetcounter{entitycount}{#1}\textbf{#2} \namedlabel{E#1}{$\langle E#1\rangle$}\\}{}

% Component
\newcounter{componentcount}
\newenvironment{component}[2]{\refsetcounter{componentcount}{#1}\textbf{Komponente \namedlabel{C#1}{$\langle C#1\rangle$}: #2}\\}{}

% Interface
\newcounter{interfacecount}
\newenvironment{interface}[2]{\refsetcounter{interfacecount}{#1}\textbf{Schnittstelle \namedlabel{I#1}{$\langle I#1\rangle$}: #2}\\}{}

% Testfall
\newcounter{testcasecount}
\newenvironment{testcase}[2]{\clearpage\refsetcounter{testcasecount}{#1}\subsection{Testfall $\langle T#1\rangle$ - #2}\label{T#1}\begin{description}\setlength{\itemsep}{-5pt}}{\end{description}}


